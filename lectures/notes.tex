\documentclass{article}
\usepackage[margin=1.0in]{geometry}
\usepackage{amsmath, amssymb, mathrsfs}
\usepackage[english]{babel}
\usepackage{graphicx}
\usepackage{enumerate}
\usepackage{tikz}



\begin{document}

\section*{Propositions}

\textit{A proposition is either True or False}

\begin{itemize}
\item may be easy or difficult to assign truth value to proposition
\item prop itself should always be precise; unambiguous
\end{itemize}

\subsection*{connectors}

\begin{align}
\text{NOT: }  \neg p &\equiv \text{ it is not the case that p} \\
\text{AND: } p \wedge q &\equiv \text{ p and q} \\
\text{OR: } p \vee q &\equiv \text{ p or q} \\
\text{IF THEN: } p \rightarrow q &\equiv \text{ if p then q / p implies q}
\end{align}

\begin{center}
\subsection*{implication}
\begin{tabular}{c c | c}
$p$ & $q$ & $p \implies q$ \\ [0.5ex]
\hline 
T & T & T \\ 
T & F & F \\
F & T & T \\
F & F & T
\end{tabular}
\end{center}

alright guess we'll just expand the table a bit

\begin{center}
\begin{tabular}{c c | c c c c}
$p$ & $q$ & $\neg p$ & $p \wedge q$ & $p \vee q$ & $p \implies q$ \\ [0.5ex]
\hline 
T & T & F & T & T & T \\ 
T & F & F & F & T & F \\
F & T & T & F & T & T \\
F & F & T & F & F & T
\end{tabular}
\end{center}

\subsection*{POP QUIZ WOO}

$$p \equiv x > 0$$
$$q \equiv y > 1$$
$$r \equiv x < y$$


\begin{table}[h!]
\centering
\begin{tabular}{c c c | c c c}
$p$ & $q$ & $r$ & $q \wedge r$ & $p \vee (q \wedge r)$ & $p \vee q$\\ [0.5ex]
\hline 
T & T & T & T & T & T\\ 
T & T & F & F & T & T\\
F & T & T & T & T & T\\
T & F & T & F & T & T\\
T & F & F & F & T & T\\
F & F & T & F & F & F\\
F & T & F & F & F & T\\ %%% this is a bad line
F & F & F & F & F & F
\end{tabular}
\caption{Note that row 7 is not actually possible.}
\end{table}

\section*{Quantifiers}

e.g., $$\text{EVERY; A; SOME; ANY; ALL; THERE EXISTS}$$

define \textit{predicate} $P(c)$ where $$C = \{c \mid c \text{ is a car}\}$$ $$ P(c) = \text{"car c has four wheels"}$$

we write the statement "for all c in C, P(c) is true" as

$$\forall c \in C : P(c)$$

e.g., for the function $f(x) = x^2$, we can write

$$\forall x \in \mathbb{R} : f(x) \geq 0$$

\newpage

\section*{More on Proofs.}

\subsection*{Direct Proof Template for proving $p \implies q$}

\textbf{Proof.}

\begin{enumerate}
\item Start by assuming that the statement claimed in $p$ is \textbf{T} 
\item Restate your assumption in mathematical terms 
\item Use mathematical and logical derivations to relate your assumption to $q$ 
\item Argue that you have shown that $q$ must be \textbf{T} 
\item End by concluding that $q$ is \textbf{T}
\end{enumerate}

\noindent\textit{Example.}

\textbf{Thm. } If $x, y \in \mathbb{Q}$, then $x + y \in \mathbb{Q}$ \\

\textbf{Proof.} 
\begin{enumerate}
\item Assume that $x,y \in \mathbb{Q}$
\item Then there are integers $a, c$ and natural numbers $b, d$ such that $x = \frac{a}{b}$ and $y = \frac{c}{d}$
\item Then $x + y = (ad + bc)/bd$
\item Since $ad + bc \in \mathbb{Z}$ and $bd \in \mathbb{N}$, $x+y$ is rational.
\end{enumerate}

\noindent\textit{Another example.}

\textbf{Thm.} If $4^x - 1$ is divisible by 3, then $4^{x+1}$ is divisible by 3 for $x \in \mathbb{R}$.

\textbf{Proof.}

\begin{enumerate}
\item Assume that $4^x-1$ is divisible by 3.
\item So $4^x-1 = 3k$ for an integer $k$, i.e. $4^x = 3k+1$
\item Observe: $4^{x+1} = 4 \cdot 4^x$. So $$4^{x+1} = 4(3k+1) = 12k + 4$$ Then $4^{x+1}-1 = 12k + 3 = 3(4k+1)$ is a multiple of 3.
\item Since it's a multiple of 3, it must be divisible by 3.
\item ayooo $q$ is \textbf{T} ***
\end{enumerate}

*** Note that we don't actually know that $4^x -1$ is divisible by 3.

\smallskip

\noindent\textit{Exercise.}

\textbf{Theorem.} For all pairs of odd integers $m, n$, the sume $m + n$ is an even integer.

\textbf{Proof.}
\begin{enumerate}
\item Assume $m$ and $n$ are both odd.
\item aighty this means that $m = 2k + 1$ and $n = 2l + 1$.
\item adding these together, we have $$m + n = 2k + 1 + 2l + 1 = 2 + 2k + 2l = 2(k + l + 1)$$
\item since $m+n$ is a multiple of 2, it is divisible by 2 and thus an even number
\item \textbf{QED}
\end{enumerate}

\subsection*{Contraposition Template for $p \implies q$}

\textbf{Proof.}
\begin{enumerate}
\item Start by assuming that the statement claimed in $q$ is \textbf{F}
\item Restate your assumption in mathematical terms
\item Use mathematical and logical derivations to relate your assumption to $p$
\item Argue that you have shown that $p$ must be \textbf{F}
\item End by concluding that $p$ is \textbf{F}
\end{enumerate}

\noindent\textit{Example}

\noindent\textbf{Theorem.} If $x^2$ is even, then $x$ is even.

\noindent\textbf{Proof.}
\begin{enumerate}
\item Assume that $x$ is odd.
\item Then $\exists k \in \mathbb{Z} : x = 2k + 1$
\item Then $x^2 = 2(2k^2 + 2k) + 1$
\item This means $x^2$ is 1 added to a multiple of 2, so it's odd.
\item $x^2$ is odd so the proof is over lol
\end{enumerate}

\noindent\textit{Exercise}

\noindent\textbf{Theorem.} If $r$ is irrational, then $\sqrt{r}$ is irrational.

\noindent\textit{Proof.}
\begin{enumerate}
\item Let's assume $\sqrt{r}$ is rational.
\item So $\exists a,b \in Z : \sqrt{r} = \frac{a}{b}$
\item What happens when we square it? $$\sqrt{r}^2 = (\frac{a}{b})^2$$ $$r = \frac{a^2}{b^2}$$ $a$ and $b$ are both integers, so $r$ must be rational
\item So it's clear that $r$ is not irrational in this case (it is rational).
\item unnecessary restatement of concluding $p$ is \textbf{F}
\end{enumerate}

\subsection*{Equivalence: sort of a sidenote} 
\begin{center}
IF AND ONLY IF
\end{center}
$$p \iff q$$

This just means you have to prove the implication \textbf{both ways}.

\subsection*{Contradictions}

e.g.,
\begin{center}
$1=2$;     $n^2 < n \text{ for } n \in \mathbb{N}$;     $| x | <  x$;     $p \wedge \neg p$
\end{center}

Wowie these look \Large\textbf{FISHY}\normalsize don't they?

\noindent\textbf{Proof Template}
\begin{enumerate}
\item To derive a contradiction, assume that $p$ is \textbf{F}
\item Restate your assumption in mathematical terms
\item Derive a \LARGE\textbf{FISHY}\normalsize statement?a contradiction that must be false
\item Thus, the assumption in step 1 is false, and $p$ is \textbf{T}
\end{enumerate}

\noindent\textit{Exercise}

\textbf{Theorem}. Let $a, b$ be integers. Then $a^2 - 4b \neq 2$

\textbf{Proof.}
\begin{enumerate}
\item Say $a^2 - 4b = 2$
\item Then $$a^2 = 2 + 4b = 2(1+2b)$$ $$a = \sqrt{2}\sqrt{1+2b}$$
\item $\sqrt{2}$ is irrational, so $a$ must be irrational, so it's not an integer. but $a$ is an integer. \LARGE\textbf{FISHY.}\normalsize
\item alright so we must have $$a^2 -4b \neq 2$$
\end{enumerate}

\subsection*{Proofs about Sets}

Let's look at $A \cup (B \cap C) = (A \cup C) \cap (A \cup C)$

\smallskip

\def\firstcircle{(90:.6cm) circle (1.cm)}
\def\secondcircle{(210:.6cm) circle (1.cm)}
\def\thirdcircle{(330:.6cm) circle (1.cm)}

\begin{minipage}{.2\textwidth}
\begin{tikzpicture}
  \begin{scope}[fill opacity=0.5]
    \clip \firstcircle;
    \fill[cyan] \firstcircle;
  \end{scope}
  \begin{scope}[fill opacity=0.5]
    \clip \secondcircle;
    \fill[yellow] \thirdcircle;
  \end{scope}
  \draw \firstcircle node[text=black,above] {$A$};
  \draw \secondcircle node [text=black,below left] {$B$};
  \draw \thirdcircle node [text=black,below right] {$C$};
  \end{tikzpicture}
\end{minipage}
=
\begin{minipage}{.2\textwidth}
\begin{tikzpicture}
  \begin{scope}[fill opacity=0.5]
    \fill[cyan] \firstcircle;
    \fill[cyan] \thirdcircle;
  \end{scope}
  \begin{scope}[fill opacity=0.5]
    \fill[yellow] \thirdcircle;
    \fill[yellow] \secondcircle;
  \end{scope}
  \draw \firstcircle node[text=black,above] {$A$};
  \draw \secondcircle node [text=black,below left] {$B$};
  \draw \thirdcircle node [text=black,below right] {$C$};
  \end{tikzpicture}
\end{minipage}
-
\begin{minipage}{.2\textwidth}
\begin{tikzpicture}
  \begin{scope}[fill opacity=0.5, even odd rule]
    \clip \firstcircle;
    \clip \thirdcircle;
    \fill[red] \secondcircle;
  \end{scope}
  \draw \firstcircle node[text=black,above] {$A$};
  \draw \secondcircle node [text=black,below left] {$B$};
  \draw \thirdcircle node [text=black,below right] {$C$};
  \end{tikzpicture}
\end{minipage}
  
\newpage

\section*{Induction}

\textbf{Template}

\begin{enumerate}
  \item Show $P(1)$
  \item Assume $P(n)$
  \item Show $P(n) \implies P(n+1)$
\end{enumerate}

%%%%%%%%%%%%%%
% 5 Feb 2017 %
%%%%%%%%%%%%%%

\newpage

\section*{More Proof-y Things}

\subsection*{Well-Ordering Principle}

\textit{Any non-empty set of natural numbers has a minimum element.}

\smallskip

This is important because induction follows form well ordering. e.g.

Take some predicate $P(n)$. If $P(1)$, and $P(n) \implies P(n+1)$, then $P(n)$ for $n \geq 1$.

\noindent\textbf{Proof.} Suppose $P(1)$ and $P(n) \implies P(n+1)$ for $n \geq 1$.

Assume $P(n)$ false for some values of $n$, with $n*$ representing the smallest counterexample for $P(n)$. Here, $n* > 1$ because $P(1)$ is true. 

Given this assumption, $n*-1$ is not a counterexample because $n*$ is the smallest counterexample, so $P(n* -1)$ is true.

But since $P(n*-1)$ is true, we must have $P(n*-1) \implies P(n*)$. So we have a contradiction. Therefore $P(n)$ is true for all $n \geq 1$.

\noindent\textbf{An example}

$$n < 2^n \text{ for } n \geq 1$$

\noindent\textbf{Proof.}

\noindent \textit{Induction.}  

$P(1)$ is true because $1 < 2$. Assume $P(n)$ true. Then

$$n + 1 \leq n + n = 2n \leq 2 \cdot 2^n = 2^{n+1}$$

So $P(n+1)$ is true and therefore $P(n)$ is true.

\smallskip

\noindent \textit{Well-ordering}

Assume that there is an $n \geq 1$ such that $n \geq 2^n$. Let $n*$ be the minimum example of this, so $n* \geq 2^n$. 

We know $1 < 2^1$, so $n* \geq 2$, which gives $\frac{1}{2}n* \geq 1$. So

$$n* - 1 \geq n* - \frac{1}{2}n* = \frac{1}{2}n* \geq \frac{1}{2} \cdot 2^{n*} = 2^{n*-1}$$

which \large means \normalsize that $n* -1$ is a \large smaller \normalsize counterexample! ooOOoOOOO.

\bigskip

\noindent \large \textbf{Harder} \normalsize

\noindent Prove $\sum_{i=1}^{n} \frac{1}{\sqrt{i}} \leq 2n$.

\noindent \textbf{Proof.}

$P(1)$: $1 \leq 2\cdot \sqrt{1}$ is true.

Assume $P(n)$. Then for $P(n+1)$ we have

$$\sum_{i=1}^{n+1} \frac{1}{\sqrt{i}} \leq 2\sqrt{n+1}$$

We can use the assumption of $P(n)$ to rewrite this

\begin{align*}
  \sum_{i=1}^{n+1} \frac{1}{\sqrt{i}} &= \sum_{i=1}^{n} \frac{1}{\sqrt{i}} + \frac{1}{\sqrt{n+1}}\\
  &\leq 2\sqrt{n} + \frac{1}{\sqrt{n+1}}
\end{align*}

And here we use a \textit{Lemma.} $2\sqrt{n} + \frac{1}{\sqrt{n+1}} \leq 2\sqrt{n+1}$

Which we prove by contradiction:

\begin{align*}
  2\sqrt{n} + \frac{1}{\sqrt{n+1}} &> 2 \sqrt{n+1} \\
  2 \sqrt{n(n+1)} + 1 &> 2(n+1) \\
  4n(n+1) &> 4(n+1)^2 \\
  4n &> 4n + 4
\end{align*}

Wow fishy.

\textit{Back to the proof:}

\begin{align*}
  \sum_{i=1}^{n+1} \frac{1}{\sqrt{i}} &\leq 2\sqrt{n} + \frac{1}{\sqrt{n+1}} \\
  &\leq 2\sqrt{n+1}
\end{align*}

So $P(n)$ is true for all $n \geq 1$.

\bigskip

\noindent \textbf{Prove} $n^2 \leq 2^n \text{ for } n \geq 4$

$$4^2 = 16 \leq 2^4 = 16$$

Assume that $n^2 \leq 2^n$ and that $2n + 1 \leq 2^n$. Then $$(n+1)^2 = n^2 + 2n + 1 \leq 2^n + 2n + 1\leq 2^n + 2^n = 2^n+1$$

\bigskip

\subsection*{the tile problem}

Can you tile a $2^n \times 2^n$ patio missing one of the center squares, using only the corner shaped tile?

let $P(n) := $ the $2^n \times 2^n$ grid minus a center square can be $L$-tiled.

Suppose $P(n)$ is \textbf{T}. WELL. The $2^{n+1} \times 2^{n+1}$ patio can be separated into four $2^n \times 2^n$ patios.

Think about adding the center $L$ to this first. Then all four of the subtiles were/are missing a corner square. Thus we can revise the original claim to be

$Q(n) :$

\begin{centering}
  (i) the $2^n \times 2^n$ grid missing a center square can be $L$-tiled. \\
  (ii) the $2^n \times 2^n$ grid missing a corner square can be $L$-tiled.
\end{centering}

So add base cases and complete the proof.





\bigskip

\noindent \subsection*{Different Problem} $P(n) : n^3 < 2^n$ for $n \geq 10$

Suppose $P(n)$ is true. Consider $P(n+1) : (n+1)^3 < 2^{n+2}$ ??

\begin{align*}
  (n+2)^3 &= n^3 + 6n^2 + 12n +8 \\
  &< n^3 + n n^2 + n^2 n + n^3 \\
  &< 4n^3 < 4 \cdot 2^n = 2^{n+2}
\end{align*}

so $$P(n) \implies P(n+2)$$

We can have two base cases to cover all cases---$P(10)$ and $P(11)$ are both true.

\section*{THE FUNDAMENTAL THEOREM OF ARITHMETIC}

\textbf{SUPPOSE $n \geq 2$. Then (i) $n$ can be written as a product of prime factors, and (2) the representation of $n$ as a product of primes is unique.}

We could use $P(n)$: $n$ is a product of primes. But this is hard. So let's use $$Q(n) : P(2) \wedge P(3) \wedge P(4) \wedge \cdots P(n)$$

\textbf{Proof.} Q(1) claims 2 is a product of primes, which is true.

Assume that $Q(n)$ is true, so each of $2, 3, \dots , n$ are prime products. Since we know $Q(n)$, to prove $Q(n+1)$, we just need to show that $n+1$ is a product of primes. There are some possible cases here:

\begin{itemize}
  \item    $n+1$ is prime. Fin. 
  \item  $n+1$ not prime, so $n+1 = kl$ where $2 \leq k,l \leq n$ 
\end{itemize}

In the second case, we know that $P(k)$ and $P(l)$ are both true, so $k$ and $l$ are both products of primes. Thus $kl$ is a product of primes, so $n+1$ is a product of primes. $Q(n+1)$ is true for all $n \geq 2$.

\subsection*{Strong Induction}

To prove $P(n) \forall n \geq 1$ by strong induction, use induction to prove the \textit{stronger} claim that $Q(n) : $ each of $P(1), P(2), \dots , P(n)$ are true.

\bigskip

\begin{tabular}{c c  c}
$ $ & Ordinary Induction & Strong Induction\\ [0.5ex]
\hline 
  Base Case & Prove $P(1)$ & Prove $Q(1) = P(1)$  \\ 
  Induction Step & $P(n) \implies P(n+1)$ & $Q(n) = P(1) \wedge \dots \wedge P(n) \implies P(n+1)$ 
\end{tabular}
















































\end{document}
