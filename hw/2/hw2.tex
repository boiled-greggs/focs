%foc
\documentclass{article}
\usepackage[margin=1.0in]{geometry}
\usepackage{amsmath, amssymb, mathrsfs}
\usepackage[english]{babel}
\usepackage{graphicx}
\usepackage{enumerate}
\usepackage{tikz}
\usetikzlibrary{shapes,backgrounds}

\title{Foundations of Computer Science HW 2}
\author{Greg Stewart}
\date{\today}

\begin{document}

\maketitle

\section*{Q1. 3.25 (c) - (d)}

\textit{Given:} $P(x, h) = $ "Person $x$ has hair $h$" and $M(x) = $ "Hair $h$ is grey"

\subsection*{(c) \normalsize Nobody is bald.}

$$\forall x,h : P(x,h)$$

or

$$\neg \exists x,h : \neg P(x,h)$$

\subsection*{(d) \normalsize Kilam does not have all grey hair}

$$\exists x,h : \neg M(x) \wedge x = "Kilam"$$ 


\section*{Q2. 3.29 (c) - (d)}

\subsection*{(c) \normalsize $\forall x : (M(x) \rightarrow \neg F(x))$}

For any person, if they are a math major, then they are not a freshman.

\subsection*{(d) \normalsize $\neg \exists x : (M(x) \wedge \neg F(x))$}

There aren't any people that are a math major \textit{and} a freshman.



\section*{Q3. 3.36 (c) - (d)}

\subsection*{(c)} $\forall x : (\exists y : x^2 = y)$

This claim is \textbf{T} in $\mathbb{N}, \mathbb{Z}, \mathbb{Q}, \mathbb{R}$

\subsection*{(d)} $\forall y : (\exists x : x^2 = y)$

This claim is \textbf{T} in $\mathbb{R}, \mathbb{Q}$



\section*{Q4. 4.8 (c) -(d)}

\subsection*{(c)} Prove by contraposition that $2^n - 1 \text{ is prime} \implies n \text{ is prime.}$

Begin by assuming that $n$ is not prime. Then there is some way to factor $n$ such that $n = xy$. So

\begin{align*}
2^{xy} -1 &= (2^x)^y - 1 \\
&= (2^y - 1)(2^{y(x-1)} + 2^{y(x-2)} + \cdots + 2^y + 1)
\end{align*}

We know that $y > 1$ so for the first factor we know $$2^y - 1 > 1$$

But this means that $2^n - 1$ is a composite number, and thus not prime. 

Thus the implication must be true.

\subsection*{(d)} Prove by contraposition that $n^3 \text{ is odd} \rightarrow n$ is odd.

Assume that $n$ is even. Then $n = 2k$ for some $k \in \mathbb{Z}$. Then $$n^3 = (2k)^3 = 8k^3$$

But this is an even number! So the original implication must be true.


\section*{Q5. 4.10 (f) \& (h): Prove by contradiction}

\subsection*{(f)} $(x,y) \in \mathbb{Z}^2 \rightarrow x^2 - 4y - 3 \neq 0$

Assume that the implication is false. That is, $(x,y) \in \mathbb{Z}^2 \rightarrow x^2 -4y -3 = 0$. Then 

\begin{align*}
4y &= x^2 - 3 \\
y &= \frac{x^2 - 3}{4} = \frac{x^2}{4} - \frac{3}{4}
\end{align*}

It is obvious now, since the right(most) hand side of the above equation subtracts a fraction, that $y$ is not an integer, so we have a contradiction. Therefore it must be true that $$(x,y) \in \mathbb{Z}^2 \rightarrow x^2 - 4y - 3 \neq 0$$

\subsection*{(h)} $\forall (a, b, c) \in \mathbb{Z}^3 : (a^2 + b^2 = c^2) \rightarrow (a \text{ or } b \text{ is even})$.

Let's assume this is false. So $$\exists (a, b, c) \in \mathbb{Z}^3 : (a^2 + b^2 = c^2) \rightarrow (a \wedge b \text{ odd})$$

Then we can say that $a = 2k + 1$ and $b = 2l + 1$ for some integers $k$ and $l$. So 

$$(2k+1)^2 + (2l+1)^2 = c^2 = 4k^2 + 4k + 1 + 4l^2 + 4l + 1 = 4(k^2 + l^2 + k+l) + 2$$

There is no constraint on $c^2$, so either $c = 2m$ or $c = 2m+1$ for some integer $m$. This leads to either 

$c^2 = 4m^2$ or $c^2 = 4m^2 + 4m +1$. In either case, we cannot have $$a^2 + b^2 = c^2$$

because in both cases, the two sides have different remainders when divided by 4. Thus we have a

contradiction and the original statement must be true.




\section*{Q6. 4.14 (o)}

\textbf{Thm.} There exists $x,y \in \mathbb{Z}$ for which $2x^2 +5y^2 = 14$.

Suppose that there is indeed such a pair $x, y$ which satisfies the equation. Then since 14 and $2x^2$ are 

both even, $5y^2$ must also be even. Since 5 is odd, this implies that $y^2$ is even, so $y$ is even and $y = 2k$ 

for some $k \in \mathbb{Z}$. So we have

\begin{align*}
  2x^2 + 20k^2 &= 14 \\
  x^2 + 10k^2 &= 7
\end{align*}

However, $10k^2 > 7$ for $k \neq 0$, so we must have 0 for that term. Then $x^2 = 7$, which means $x \notin \mathbb{Z}$, so we 

have a contradiction. Therefore the theorem is false. 

\section*{Q7. 4.26 (c)}

\def \setA{ (0,0) circle (.75cm) }
\def \setB{ (60:.75) circle (.75cm) }
\def \setC{ (.75,0) circle (.75cm) }



  $$A \cap (B \cup C) = (A \cap B) \cup (A \cap C)$$

\begin{minipage}{.23\textwidth}

  \begin{tikzpicture}

    \begin{scope}[opacity = 0.5]
      \fill[cyan] \setA;
    \end{scope}

    \draw \setA;
    \draw \setB;
    \draw \setC;
    \draw (-.75,0) node[left] {$A$};
    \draw (.35,1.75) node {$B$};
    \draw (1.5,0) node[right] {$C$};

  \end{tikzpicture}

\end{minipage}
$\cap$
\begin{minipage}{.23\textwidth}

  \begin{tikzpicture}

    \begin{scope}[opacity = 0.5]
      \fill[cyan] \setB;
      \fill[cyan] \setC;
    \end{scope}

    \draw \setA;
    \draw \setB;
    \draw \setC;
    \draw (-.75,0) node[left] {$A$};
    \draw (.35,1.75) node {$B$};
    \draw (1.5,0) node[right] {$C$};

  \end{tikzpicture}

\end{minipage}
=
\begin{minipage}{.23\textwidth}

  \begin{tikzpicture}

    \begin{scope}[opacity = 0.5]
      \clip \setA;
      \fill[cyan] \setB;
    \end{scope}

    \begin{scope}[opacity = 0.5]
      \clip \setC;
      \fill[cyan] \setA;
    \end{scope}

    \draw \setA;
    \draw \setB;
    \draw \setC;
    \draw (-.75,0) node[left] {$A$};
    \draw (.35,1.75) node {$B$};
    \draw (1.5,0) node[right] {$C$};

  \end{tikzpicture}

\end{minipage} \\ \\ \\
$\null \qquad\qquad\ \ A \qquad\qquad\  \ \ \qquad\qquad\qquad(B \cup C)$
 \\ \\ \\

\begin{minipage}{.23\textwidth}

  \begin{tikzpicture}

    \begin{scope}[opacity = 0.5]
      \clip \setA;
      \fill[magenta] \setB;
    \end{scope}

    \draw \setA;
    \draw \setB;
    \draw \setC;
    \draw (-.75,0) node[left] {$A$};
    \draw (.35,1.75) node {$B$};
    \draw (1.5,0) node[right] {$C$};

  \end{tikzpicture}

\end{minipage}
$\cup$
\begin{minipage}{.23\textwidth}

  \begin{tikzpicture}

    \begin{scope}[opacity = 0.5]
      \clip \setA;
      \fill[magenta] \setC;
    \end{scope}

    \draw \setA;
    \draw \setB;
    \draw \setC;
    \draw (-.75,0) node[left] {$A$};
    \draw (.35,1.75) node {$B$};
    \draw (1.5,0) node[right] {$C$};

  \end{tikzpicture}

\end{minipage}
=
\begin{minipage}{.23\textwidth}

  \begin{tikzpicture}

    \begin{scope}[opacity = 0.5]
      \clip \setA;
      \fill[magenta] \setC;
    \end{scope}

    \begin{scope}[opacity = 0.5]
      \clip \setA;
      \fill[magenta] \setB;
    \end{scope}

    \draw \setA;
    \draw \setB;
    \draw \setC;
    \draw (-.75,0) node[left] {$A$};
    \draw (.35,1.75) node {$B$};
    \draw (1.5,0) node[right] {$C$};

  \end{tikzpicture}

\end{minipage}  \\ \\ \\
$\null \qquad\qquad (A \cap B) \qquad\qquad\qquad \  \ \ \qquad(A \cup C)$ \\



\section*{Q8. 4.27 (c)}

Prove that $A \cap (B \cup C) = (A \cap B) \cup (A \cap C)$

Let there be $x \in A \cap (B \cup C)$ so that $x \in A$ and $x \in B \vee C$. So $x \in A \wedge B$ or $x \in A \wedge C$. So $x \in (A \cap B) \cup (A \cap C)$. Thus we have 

$$A \cap (B \cup C) \subseteq (A \cap B) \cup (A \cap C)$$

Now the other way. Let there be $y \in (A \cap B) \cup (A \cap C)$. So $y \in A \cap B$ or $y \in A \cap C$. In either case, $y \in A$, and either $y \in B$ or $y \in C$, so $y \in B \cup C$. Then $y \in A \cap (B \cup C)$. Therefore

$$(A \cap B) \cup (A \cap C) \subseteq A \cap (B \cup C)$$

Thus the equality is true.























\end{document}
