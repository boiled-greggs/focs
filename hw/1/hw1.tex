\documentclass{article}
\usepackage[margin=1.0in]{geometry}
\usepackage{amsmath, amssymb, mathrsfs}
\usepackage[english]{babel}
\usepackage{graphicx}

\title{Foundations of Computer Science HW 1}
\author{Greg Stewart}
\date{\today}

\begin{document}

\maketitle

\section*{Q1}

(c) $A = \{x \mid x^2 = 6 : x \in \mathbb{Z} \}$ $$A = \{-3, 3\}$$

\noindent (d) $A = \{x \mid x = x^2-1 : x \in \mathbb{R} \}$ $$A = \Big\{ \frac{1 - \sqrt{5}}{2}, \frac{1+\sqrt{5}}{2} \Big\} $$

%%%%%%%%%%%%%%%%%%%%%%

\section*{Q2}

(c) $C = \{1, 2, 4, 7, 11, 16, 22, \dots \}$ $$C = \big\{x \mid x = \frac{n(n-1)}{2} + 1 : n \in \mathbb{N}\big\}$$

\noindent (d) $D = \{ \dots , \frac{1}{8}, \frac{1}{4}, \frac{1}{2}, 1, 2, 4, 8, \dots \}$ $$D = \{ x \mid x = 2^n : n \in \mathbb{Z} \}$$


%%%%%%%%%%%%%%%%%%%%%%%%%%%%%

\section*{Q3}

$B = \{\{a,b\}, a, b, c\}$. What is the power set $\mathscr{P}(B)$

\begin{align*}
\mathscr{P} = &\Big\{ \{\}, \{a\}, \{b\}, \{c\}, \big\{a,b\}\big\}, \{a, b\}, \{a,c\}, \big\{a,\{a,b\}\big\}, \{b,c\}, \big\{b, \{a,b\}\big\},  \\
&\big\{c, \{a,b\}\big\}, \{a,b,c\}, \big\{a, b, \{a,b\}\big\}, \big\{a, c, \{a,b\}\big\}, \big\{b, c, \{a,b\}\big\}, \big\{a,b,c,\{a,b\}\big\}\Big\}
\end{align*}

%%%%%%%%%%%%%%%%%%%%%%%%%%%%%

\section*{Q4}

\textit{Problem.} List all subsets of $\{a, b, c, d\}$ that contain $c$ but not $d$.

The subsets are:

$$\{c\}, \{a, c\}, \{b, c\}, \{a, b, c\}$$

%%%%%%%%%%%%%%%%%%%%%%%%%%%%%

\section*{Q5}

$|A| = 7 \text{ and } |B| = 4$. 

\smallskip

$0 \leq |A \cap B| \leq 4$ because it is possible that none of the elements of $A$ and $B$ are the same, and as many 

as all 4 elements of $B$ could also be in $A$.

\smallskip

$7 \leq |A \cup B| \leq 11$ because in the case where all 4 elements of $B$ are in $A$, the union of the sets is just $A$. 

However, as many as 4 new elements from $B$ could be added to $A$ to form the union, if they do not 

already exist in $A$.


%%%%%%%%%%%%%%%%%%%%%%%%%%%%%

\section*{Q6}

\textbf{Fact.} $\sqrt{3}$ is irrational.

\smallskip

\noindent \textbf{Proof.}

Let's assume that $\sqrt{3}$ is rational. Then we can write $$\sqrt{3} = \Big\{ \frac{a_1}{b_1}, \frac{a_2}{b_2}, \frac{a_3}{b_3}, \cdots \Big\}$$

where $a_i, b_i \in \mathbb{N}.$ We use the well ordering principle to assert that there is a minimum $b_i$, which we will 

call $b$. There is of course an $a_i$ corresponding to $b$, and which we will call $a$. Since $b$ is the minimum 

possible value for the denominator, it is true that $a$ and $b$ have no common factor. So let's write our 

expression for $\sqrt{3}$ and solve for $a$.

\begin{align*}
\sqrt{3} &= \frac{a}{b} \\
a^2 &= 3b^2
\end{align*}

We don't actually have to solve all the way. What we can see here is that either both $a$ and $b$ are odd, 

or they are both even. Let's say they're both odd. Then

\begin{align*}
a &= 2n + 1 \\
b &= 2m + 1 \\
\end{align*}

where $n, m \in \mathbb{N}$. Substituting this into our previous expression, we have

\begin{align*}
(2n+1)^2 &= 3(2m+1)^2 \\
4n^2 + 4n + 1 &= 3(4m^2 + 4m + 1) \\
2n^2 + 2n &= 6m^2 + 6m + 1 \\
2(n^2 + n) &= 2(3m^2 + 3m) +1
\end{align*}

Now we can see that the LHS of the expression is even since it is a multiple of 2, and the RHS is odd. 

They must not be the same, so we have a contradiction. Therefore, there are no values of $a$ and $b$ which 

give $\sqrt{3}$,  so $\sqrt{3}$ \textbf{is irrational}.


\bigskip

\noindent\textit{Problem.} Try to prove that $\sqrt{9}$ is irrational this way.

Assume that $\sqrt{9}$ is rational. As before, $$\sqrt{9} = \Big\{ \frac{a_1}{b_1}, \frac{a_2}{b_2}, \frac{a_3}{b_3}, \cdots \Big\}$$

is the set of possible representations, with $a_i, b_i \in \mathbb{N}.$ There is a minimum $b_i$ called $b$ and 

corresponding $a$. These have no common factor. We can again look at solutions for $a$.

\begin{align*}
\sqrt{9} &= \frac{a}{b} \\ 
a^2 &= 9b^2
\end{align*}

Again we can use the facts from the previous proof to expand both sides of the expression.

\begin{align*}
(2n+1)^2 &= 9(2m+1)^2 \\
4n^2 + 4n + 1 &= 9(4m^2 + 4m + 1) \\ %36mm + 36m + 9
2n^2 + 2n &= 18m^2 + 18m + 4 \\
2(n^2 + n) &= 2(9m^2 + 9m + 2)
\end{align*}

But this time, both the LHS and RHS are even, and nothing about our restrictions for a and b are 

violated. Since there is no contradiction here, we cannot conclude that $\sqrt{9}$ is irrational.


 


%%%%%%%%%%%%%%%%%%%%%%%%%%%%%

\section*{Q7}

\begin{table}[h!]
\centering
\begin{tabular}{c c c | c}
$p$ & $q$ & $r$ & $\neg(p \wedge r) \wedge q$ \\
\hline 
T & T & T & F\\ 
T & T & F & T\\
T & F & T & F\\
T & F & F & F\\
F & T & T & T\\
F & T & F & T\\
F & F & T & F\\
F & F & F & F
\end{tabular}
\end{table}



%%%%%%%%%%%%%%%%%%%%%%%%%%%%%





\end{document}