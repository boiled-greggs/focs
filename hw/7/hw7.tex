\documentclass{article}
\usepackage[margin=1.0in]{geometry}
\usepackage{amsmath, amssymb, mathrsfs}
\usepackage{mathtools}
\usepackage[english]{babel}
\usepackage{graphicx}
\usepackage{enumerate}
\usepackage{tikz}
\usepackage{ wasysym }
\usetikzlibrary{shapes,backgrounds}
\DeclarePairedDelimiter\Floor\lfloor\rfloor
\DeclarePairedDelimiter\Ceil\lceil\rceil

\title{Foundations of Computer Science HW 7}
\author{Greg Stewart}
\date{\today}

\begin{document}

\maketitle

\section*{Q1. \normalsize Count.}

\begin{enumerate}[(a)]\setcounter{enumi}{4}
  \item The number of orders in which a traveling salesman can visit the 50 states.
    
    $50! \approx 3.04141 \times 10^{64}$
  \item The number of poker hands with a card in every suit.
    
    The number of hands with a card in every suit is given by
    \begin{align*}
      \begin{pmatrix}
        4 \\ 1 
      \end{pmatrix}
      \begin{pmatrix}
        13 \\ 2 
      \end{pmatrix}
      \begin{pmatrix}
        13 \\ 1 
      \end{pmatrix}
      \begin{pmatrix}
        13 \\ 1 
      \end{pmatrix}
      \begin{pmatrix}
        13 \\ 1 
      \end{pmatrix}
      &= \frac{4!}{1!(4-1)!}\frac{13!}{2!(13-2)!}\frac{13!}{1!(13-1)!}\cdot3 \\
      &= 12168
    \end{align*}
    
\end{enumerate}



\section*{Q2. \normalsize What's the coefficient of $x^3$?}

\begin{enumerate}[(a)]\setcounter{enumi}{2}
  \item $(2 - 2x)^6$

    From the Pascal's triangle pattern, the coefficient for the $x^3$ term is 20, for $(x+1)^6$. Since in this case $x^3$ is the middle term, 2 is multiplied out 6 times, and it will be negative, so we have for the coefficient $$-2^6\cdot20 = -1280$$


\end{enumerate}



\section*{Q3. \normalsize Sets A, B, C have sizes 2, 3, 4. What are the min and max for $|A\cup B\cup C|$}

If $A \subseteq B \subseteq C$, then we have $\min|A\cup B\cup C| = 4$, since the size of $C$ is 4.

\noindent If all the sets are distinct and no elements from any set are in any other, then we add the sizes together, so $\max|A\cup B\cup C| = 2 + 3 + 4 = 9$.


\section*{Q4. \normalsize Consider the binary strings consisting of 10 bits.}

\begin{enumerate}[(a)]
    \item How many contain fewer 1's than 0's?

      This is all the strings with $\leq$ 4 1's. Consider that the number of strings with fewer 1's than 0's is the same as the number of strings with fewer 0's than 1's. Now, there are 

      $$2^10 = 1024$$

      possible strings. If we subtract from this the number of strings with an equal number of 1's and 0's, and take half that result, we'll have the answer:

      $$\frac{1}{2}(1024 - \frac{10!}{5!5!}) = \frac{1}{2}(1024-252) = 386$$

      So the answer is 386.
\end{enumerate}



\section*{Q5. \normalsize In each case, count the number of objects/arrangements of given type:}

\begin{enumerate}[(a)]\setcounter{enumi}{5}
  \item US SSN's with digits stricty increasing order.
    
    The 10 digits in strictly increasing order is 0123456789. This is of course 10 digits long, so removing any one digit from this will leave us with a SSN in strictly increasing order. Since there are 10 digits, this means there are exactly 10 SSNs with strictly increasing digits.
  \item US SSN's with digits in non-decreasing order.

    There are 10 possible digits and an SSN is 9 digits long, so we can evaluate as follows: 

    $$10\cdot\prod_{i=1}^{8} \frac{10+i}{i+1} = 10\cdot(\frac{11}{2}\cdot\frac{12}{3}\cdots\frac{18}{9}) = 10\cdot4862 = 48620$$
\end{enumerate}



\section*{Q6. \normalsize Use inclusion-exclusion to count the integer solutions to $x_1 + x_2 + x_3 = 20$ where $-2 \leq x_1 \leq 10$, $2 \leq x_2 \leq 8$, and $0 \leq x_3 \leq 15$.}

Let

\begin{align*}
  y_1 &= x_1 + 2 \qquad 0 \leq y_1 \leq 12 \\
  y_2 &= x_2 - 2 \qquad 0 \leq y_2 \leq 6 \\
  y_3 &= x_3 \qquad \quad \quad 0 \leq y_3 \leq 15
\end{align*}

So we must find the number of solution to $$y_1 + y_2 + y_3 = 20.$$  

Let $A_1$ be the solution for $y_1 \geq 13$, $A_2$ the set for $y_2 \geq 7$, and $A_3$ the set for $y_3 \geq 16$. The total number of non negative solutions without restriction is 

$$|S| = 
  \begin{pmatrix}
    20 + 3 - 1 \\ 3 - 1 \\ 
  \end{pmatrix}
  \begin{pmatrix}
    22 \\ 2 \\ 
  \end{pmatrix}
  = \frac{22!}{2!20!} = 231
$$

And then for the sizes of the $A_i$ we have

\begin{align*}
  |A_1| &= 
  \begin{pmatrix}
    7 + 3 - 1 \\ 2 \\ 
  \end{pmatrix}
  = 36 \\
  |A_2| &=
  \begin{pmatrix}
    13 + 3 - 1 \\ 2 \\ 
  \end{pmatrix}
  = 105 \\
  |A_3| &=
  \begin{pmatrix}
    4 + 3 - 1 \\ 2 \\ 
  \end{pmatrix}
  = 15 \\
\end{align*}

And for the sizes of their intersections,

\begin{align*}
  |A_1 \cap A_2| &= 
  \begin{pmatrix}
    20 - 13 - 7 + 2 \\ 2 \\ 
  \end{pmatrix}
  = 1 \\
  |A_1 \cap A_3| &=
  \begin{pmatrix}
    20 - 13 - 16 + 2 \\ 2 \\ 
  \end{pmatrix}
  = 0 \\
  |A_2 \cap A_3| &=
  \begin{pmatrix}
    20 - 7 - 16 + 2 \\ 2 \\ 
  \end{pmatrix}
  = 0 \\
\end{align*}

So the total number of solutions is 

$$231 - (36 + 105 + 15) + (1 + 0 + 0) = 76$$



\section*{Q7. \normalsize 8 distinguishable dice are rolled. How many outcomes are there?}

The number of outcomes is given by $6^8 = 1679616$ outcomes.

\begin{enumerate}[(a)]
  \item How many outcomes do not contain a 1? How many do not contain a 1 or 2?

    \textbf{No 1.} This leaves 5 possibilities for each roll, so $$5^8 = 390625$$

    \textbf{No 1 or 2.} This leaves 4 possibilities for each roll, so $$4^8 = 65536$$
    
  \item How many outcomes contain all 6 numbers?

    From inclusion/exclusion, this is obtained by

    \[
      \begin{pmatrix}
        6 \\ 6
      \end{pmatrix} \cdot 6^8 -
      \begin{pmatrix}
        6 \\ 5
      \end{pmatrix} \cdot 5^8 +
      \begin{pmatrix}
        6 \\ 4
      \end{pmatrix} \cdot 4^8 -
      \begin{pmatrix}
        6 \\ 3
      \end{pmatrix} \cdot 3^8 +
      \begin{pmatrix}
        6 \\ 2
      \end{pmatrix} \cdot 2^8 -
      \begin{pmatrix}
        6 \\ 1
      \end{pmatrix} \cdot 1^8
      = 191520
    \]
\end{enumerate}



\section*{Q8. \normalsize Randomly roll two dice. Compute the probabilities to roll:}

\begin{enumerate}[(a)]
  \item One six.

    This is given by $$2\cdot\frac{1}{6}\cdot \frac{5}{6} = \frac{10}{36}$$
  \item A sum of 6.
    
    For rolling the numbers 1, 2, 3, 4, and 5, there is exactly one roll for the second die that adds to 6, so the answer is $$5\cdot \frac{1}{6}\frac{1}{6} = \frac{5}{36}$$
  \item A sum that is a multiple of 3.

    The sums which are multiples of 3 are 3, 6, 9, and 12. P(3) is $\frac{2}{36}$, P(6) is $\frac{5}{36}$, P(9) is $\frac{4}{36}$, and P(12) is $\frac{1}{36}$, so we have

    $$P(3\mid sum) = \frac{2}{36} + \frac{5}{36} + \frac{4}{36} + \frac{1}{36} = \frac{7}{18}$$
\end{enumerate}



\section*{Q9. \normalsize Draw two cards randomly from a 52-card deck. Compute probabilities:}

\begin{enumerate}[(a)]
  \item The first is a $K$ and the second a picture-card---a (f)ace.

    There are 4 kings and 16 face cards, but only picture cards remain after drawing the king, so we have

    $$\frac{4}{52}\cdot\frac{16-1}{51} = \frac{5}{221}$$
  \item At least one card is a picture-card.

    We need the probability that we draw exactly one picture card, plus the probability of drawing two. So we get:

    $$\Big(\frac{16}{52}\frac{36}{51} + \frac{36}{52}\frac{16}{51}\Big) + \frac{16}{52}\frac{15}{51} = \frac{116}{221}$$
\end{enumerate}



\section*{Q10. \normalsize One in 20 men are colorblind and one in 400 women are colorblind. There are an equal number of men and women. You select a person at random.}

\begin{enumerate}[(a)]
  \item What is the probability that the person is colorblind?

    The probabilities of selecting a woman or man are equal, so we have

    $$P(colorblind) = \frac{1}{2}\frac{1}{400} + \frac{1}{2}\frac{1}{20} = \frac{21}{800}$$
  \item The person is colorblind. What is the probability that the person is male?

    $$P(Male \mid colorblind) = \frac{P(Male)\cdot P(colorblind \mid Male)}{P(colorblind)} = \frac{1/2 \cdot 1/20}{\frac{21}{800}} = \frac{20}{21} \approx .952$$
\end{enumerate}




\end{document}
