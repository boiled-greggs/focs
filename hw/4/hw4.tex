\documentclass{article}
\usepackage[margin=1.0in]{geometry}
\usepackage{amsmath, amssymb, mathrsfs}
\usepackage{mathtools}
\usepackage[english]{babel}
\usepackage{graphicx}
\usepackage{enumerate}
\usepackage{tikz}
\usetikzlibrary{shapes,backgrounds}
\DeclarePairedDelimiter\Floor\lfloor\rfloor
\DeclarePairedDelimiter\Ceil\lceil\rceil

\title{Foundations of Computer Science HW 4}
\author{Greg Stewart}
\date{\today}

\begin{document}

\maketitle

\section*{Q1) 6.4}

\textit{Problem.} Prove $P(n): \quad n^7 < 2^n$ for $n \geq 37$.

\noindent (a) with induction.

\textbf{Proof.} Take the base case, $n = 37$. $37^7 = 94931877133 < 2^37 = 137438953472$.

Assume that for some $n$, $P(n)$ is true. So for $P(n+1)$ we have

\begin{align*}
  (n+1)^7 &= n^7 + 7n^6 + 21n^5 + 35n^4 + 35n^3 + 21 n^2 + 7n + 1 \\
  &< 2^n + 7n^6 + 21n^5 + 35n^4 + 35n^3 + 21 n^2 + 7n + 1 
  &< 2^n + 2^n = 2^{n+1}
\end{align*}

In order for this to be true, we must show $7n^6 + 21n^5 + 35n^4 + 35n^3 + 21 n^2 + 7n + 1 < 2^n$ in essentially the same manner. The base case $n = 37$ is true, so now for $n+1$:

$$  7(n+1)^6 + 21(n+1)^5 + 35(n+1)^4 + 35(n+1)^3 + 21 (n+1)^2 + 7(n+1) + 1 < $$
\begin{align*}
  &< 2^n + 42 n^5 + 210 n^4 + 490 n^3 + 630 n^2 + 434 n + 126 \\
  &< 2^{n+1}
\end{align*}

But now we have to prove it again for the 5-th degree polynomial. We have to do this over and over down to the 1st degree, so I'll just show how this goes:

\begin{align*}
  42 (n+1)^5 + 210 (n+1)^4 + 490 (n+1)^3 + 630 (n+1)^2 + 434 (n+1) + 126 &< \\
  &< 2^n + 210 n^4 + 1260 n^3 + 3150 n^2 + 3780 n + 1806 \\
  &< 2^{n+1}
\end{align*}

\begin{align*}
  210 (n+1)^4 + 1260 (n+1)^3 + 3150 (n+1)^2 + 3780 (n+1) + 1806 &< 2^n + 840 n^3 + 5040 n^2 + 10920 n + 8400\\
  &< 2^{n+1}
\end{align*}

\begin{align*}
  840 (n+1)^3 + 5040 (n+1)^2 + 10920 (n+1) + 8400 &< 2^n + 2520 n^2 + 12600 n + 16800\\
  &< 2^{n+1}
\end{align*}

\begin{align*}
  2520 (n+1)^2 + 12600 (n+1) + 16800 &< 2^n + 5040 n + 15120\\
  &< 2^{n+1}
\end{align*}

\begin{align*}
  5040 n + 15120 &< 2^n + 5040 \\
  &< 2^{n+1}
\end{align*}

where $5040 < 2^n$ for $n\geq37$. Since we have shown all these in sequence, we see that it's true that $n^7 < 2^n$ for $n\geq 37$. $P(n)$ is confirmed.
    
    


\noindent (b) with leaping induction.

\textbf{Proof.} Take the base case, $n = 37$. $37^7 = 94931877133 < 2^37 = 137438953472$.

Assume that for some $n$, $n^7 < 2^n$. Then for $n+3$ we have

\begin{align*}
  (n+3)^7 &= n^7 + 21n^6 + 189n^5 + 945n^4 + 2835n^3 + 5103n^2 + 5103n + 2187 \\
  &< n^7 + n\cdot n^6 + n^2\cdot n^5 + n^3\cdot n^4 + n^4\cdot n^3 + n^5\cdot n^2 + n^6\cdot n + n^7 \\
  &< 8\cdot n^7 \\
  &< 8\cdot 2^n = 2^{n+3}
\end{align*}

where the step made in the second line, where coefficients are substituted out for powers of $n$, is true in all cases for $n \geq 37$. 

So for the predicate $P(n) = n^7 < 2^n$ for $n \geq 37$, we have that $P(n) \implies P(n+3)$. This is now just a leaping induction problem, and we can demonstrate a couple more base cases to ensure all $n$ are covered:

\smallskip

$P(37) \implies P(40)$

$P(38)$: $38^7 = 114415582592 < 2^{38} = 274877906944$ so $P(38) \implies P(41)$.

$P(39)$: $39^7 = 137231006679 < 2^{39} = 549755813888$ so $P(39) \implies P(42)$

\smallskip

Since $P(n) \implies P(n+3)$ and we have shown the first three base cases are true, the claim is true for all $n \geq 37$. 




\section*{Q2) 6.16}
\textit{Problem.} Prove that, for all $n \geq 1$, there is $k \geq 0$ and $l$ odd such that $n = 2^k l$.

%2 = 1 * 2^1
%4 = 1 * 2^2
%6 = 3 * 2^1
%8 = 1 * 2^3
%10 = 5 * 2^1
%12 = 3 * 2^2
%14 = 7 * 2^1
%16 = 1 * 2^4

Consider all odd numbers greater than 1---all of that can be written as $2^k l$, where $k = 0$ and $l$ is the odd number itself. This is trivial; we just need to show that all even numbers greater than 1 can be written in this form. Consider $n = 2$. This can be written as $2^1 \cdot 1 = 2$, which is the correct form. It is also a power of $2$, and, in fact, all powers of 2 can easily be written in this form, as $1*2^k$, since 1 is odd. Now, the only case left is when the even $n$ is not a power of 2. If $n/2$ is odd, then $n = 2^1 \cdot \frac{n}{2}$, which is the right form. We can prove the rest by induction, and we have already seen the base case.

Let's assume that all values up to $n$ can be written in the form $n = 2^k l$. Then when $n/2$ is even, $n = \frac{n}{2} \cdot 2^1$, and by the induction hypothesis we know that $n/2$ can be written in this form, so the formula for $n/2$ need only be double, or multiplied by $2^1$, to obtain $n$. Thus, $n$ also satisfies this form.

So we have shown that $\exists k \geq 0$ s.t. $n = 2^k l \forall n \geq 1$.


\section*{Q3) 7.4(c)}
\textit{Problem.} Guess a formula for $A_n$ and prove it by induction.

$$A_0 = 1; A_1 = 2; A_n = 2A_{n-1} - A_{n-2} + 2 \qquad n\geq 2$$

Let's look at some cases.

\begin{align*}
  A_2 &= 2(2) - 1 + 2 = 5 \\
  A_3 &= 2(5) - 2 + 2 = 10 \\
  A_4 &= 2(10) - 5 + 2 = 17 \\
  A_5 &= 2(17) - 10 + 2 = 26 \\
  A_6 &= 2(26) - 17 + 2 = 37
\end{align*}

Based on these results, the formula for $A_n$ appears to be $$A_n = n^2 + 1$$

We prove this by induction.

The base case here is $2^2 + 1 = 5$, which we know to be true. Let's assume that for all k such that $2 \leq k \leq n$, $A_k = k^2 + 1 = 2A_{k-1} - A_{k-2} + 2$. Then we have

\begin{align*}
  A_{k+1} &= 2A_k - A_{k-1} + 2 \\
  &= 2(k^2+1) - (k-1)^2 + 1 \\
  &= 2k^2 + 2 - k^2 + 2k - 1 + 1 \\
  &= k^2 + 2k + 2 \\
  &= (k+1)^2 + 1
\end{align*}

Which is exactly what we were looking for the the $k+1$ case! Thus this formula is correct:

$$A_n = n^2 + 1$$

\section*{Q4) 7.6}
\textit{Problem.} Use the function $f(n)$ as defined here, where $n \in \mathbb{N}$

\[
  f(n) =
  \begin{cases}
    0 & n = 1 \\
    f(\Floor{n/2}) + f(\Ceil{n/2}) + 1 & n > 1
  \end{cases}
\]

\noindent\textbf{(a)} Is $f$ a well-defined function?

$f$ is well-defined as it is not ambiguously defined for any values of $n$. That is, it is single-valued for all $n \in \mathbb{N}$. This comes from the fact that the function is always increasing.

\smallskip

\noindent\textbf{(b)} Tinker and guess a formula for $f(n)$.

Let's take a look at some cases for $n > 1$:

\begin{align*}
  f(2) &= 0 + 0 + 1 = 1 \\
  f(3) &= 0 + 1 + 2 = 2 \\
  f(4) &= 1 + 1 + 1 = 3 \\
  f(5) &= 1 + 2 + 1 = 4 \\
  f(6) &= 2 + 2 + 1 = 5
\end{align*}

From this is seems obvious that the formula is $$f(n) = n - 1$$

\textbf{(c)} Prove your guess.

Take the base case, $n = 1$: $f(1) = 1 - 1 = 0$, which we know to be true.

For the $n$ case, assume that $f(n) = n - 1$ is true, for all $n$ from $0$ to $n$. Then we have two possibilities for $n+1$: either $n+1$ is even or odd.

if $n+1$ is even, then 

\begin{align*}
  f(n+1) &= f(\frac{n+1}{2}) + f(\frac{n+1}{2}) + 1 \\
  &= \frac{n+1}{2} - 1 + \frac{n+1}{2} - 1 + 1 \\
  &= \frac{2n + 2}{2} -1 \\
  &= n
\end{align*}

which is exactly what we're looking for; and if $n + 1$ is odd, then

\begin{align*}
  f(n+1) &= f(\frac{n}{2}) + f(\frac{n+2}{2}) + 1 \\
  &= \frac{n}{2} - 1 + \frac{n+2}{2} - 1 + 1 \\
  &= \frac{2n + 2}{2} - 1 \\
  &= n
\end{align*}

which is again exactly what we want for the $n+1$ case! Thus we have shown that this formula is correct:

$$f(n) = n-1$$

\section*{Q5) 7.8(o)}
\textit{Problem.} Prove $P(n): \quad F_n^2 + F_{n+1}^2 = F_{2n+1}$.

Note that $F_n = F_{n-1} + F_{n-2}$

$$F_1^2 + F_2^2 = 1 + 1 = 2 = F_3$$

Let's assume that $P(1) \wedge P(2) \wedge \dots \wedge P(n-1) \wedge P(n)$ are all true. For $P(n+1)$, we have

\begin{align*}
  F_{2n+3} &= F_{n+1}^2 + F_{n+2}^2 \\
  &= (F_n + F_{n-1})^2 + (F_{n+1} + F_n)^2 \\
  &= F_n^2 + F_{n-1}^2 + 2F_n F_{n-1} + F_{n+1}^2 + F_n^2 + 2F_{n+1} F_n \\
  &= F_{n+1}^2 + 2F_n^2 + F_{n-1}^2 + 2F_n (F_{n+1} + F_{n-1}) \\
  &= F_{2n+1} + F_n^2 + F_{n-1}^2 + 2F_n (F_{n+1} + F_{n-1})
\end{align*}

$F_n F_{n-1} + F_n F_{n+1} = F_{2n}$, so 

\begin{align*}
  F_{2n+3}&= F_{2n+1} + F_n^2 + F_{n-1}^2 + 2F_{2n} \\
  &= F_{2n+1} + F_{2n-1} + 2F_{2n} \\
  &= F_{2n+2} + F_{2n+1}
\end{align*}

This is exactly the formula for the Fibonacci number $F_{2n+3}$, so we have shown $P(n)$ to be true!


\section*{Q6) 7.26}
\textit{Problem.} Give pseudocode for a recursive function that computes $3^{2^n}$ on input $n$.

%3^1 = 3    n = 0   3^(2^0)
%3^2 = 9    n = 1   3^(2^1)
%3^4 = 81   n = 2   3^(2^2)
%3^8 = 6561 n = 3   3^(2^3)

\begin{verbatim}
function powers(sq, n):
  input: n, the number to which you raise 3^2
         sq, a running square of 3 (exponentiation by squaring)
          when first called, sq = 3
  output: 3^2^n

  if n = 0: return sq
  return powers(sq*sq, n-1)
\end{verbatim}

\noindent\textbf{(a)} Prove that your function correctly computes $3^{2^n}$ for every $n \geq 0$.

This is a simplified exponentiation by squaring algorithm. We have to account for fewer cases because 3 is always raised to a power of 2.

Consider what the algorithm is doing for each recurrence: squaring the value of $sq$. Starting with 3, this means we have the current value being squared each time in this sequence:

$$3 \implies 3^2 \implies (3^2)^2 = 3^4 \implies (3^4)^2 = 3^8 \implies \cdots \implies 3^{2^n}$$

Every time the function recurses, the number is squared again, which means the exponent is doubled. The exponent is thus doubled $n$ times, making the exponent by definition $2^n$. This is exactly what we want: $$3^{2^n}$$


\noindent\textbf{(b)} Obtain a recurrence for the runtime $T_n$. Guess and prove a formula for $T_n$.

Each recurrence does one multiplication, which we will call $O(1)$, and this occurs $n$ times, so the recurrence in the runtime is $$T_n = T_{n-1} + O(1)$$

Obviously, this makes the formula for $T_n$ simply $T_n = n$, as each recurrence reduces $n$ by 1, down to the $n = 0$ case, and only one computation is done per recurrence. 


\section*{Q7) 7.31(c)}
\textit{Problem.} Give recursive definition for the set $S$ in each of the following cases.

$$S = \{\text{all strings with the same number of 0's and 1's}\}$$

Base case: $x = \epsilon$, where $\epsilon$ is just an empty string.

Rest of the set: 

$x \in S \implies (0x1 \wedge 1x0) \in S$

\section*{Q8) 8.7(a)-(c)}
\textit{Problem.} The set $P$ of parenthesis strings has defintion

\begin{align*}
  \epsilon \in P \\
  x \in P \implies [x] \in P \\
  x,y \in P \implies xy \in P
\end{align*}

\noindent\textbf{(a)} Which of $[[[ ]]]][$, $[ ][[ ]][[ ]]$ and $[[ ][ ][ ]$ are in $P$? Give derivations.


Only $[ ][[ ]][[ ]]$ is in $P$. This is derived in this order:

$$x = \epsilon \implies [] \implies [[]]$$ 
$$y = \epsilon \implies [] \implies [[]]$$
$$x,y \implies [[]][[]]$$
$$z = \epsilon \implies []$$
$$z, [[]][[]] \implies [][[]][[]]$$

\noindent\textbf{(b)} Give two different derivations of $[ ][ ][[ ]]$

\noindent First:
$$x = \epsilon \implies [] \implies [[]]$$
$$y = \epsilon \implies []$$
$$x,y \implies [][[]]$$
$$x, [][[]] \implies [][][[]]$$

\noindent Second:
$$x = \epsilon \implies []$$
$$y = \epsilon \implies []$$
$$x,y \implies [][]$$
$$z = \epsilon \implies [] \implies [[]]$$
$$[][], z \implies [][][[]]$$


\noindent\textbf{(c)} Prove by structural induction that every string in $P$ has even length.

For the empty string $\epsilon$ the length $l_{\epsilon} = 0$, which is even. For the next case, where we get $x = []$, the length is $l_1 = 2$, which is also even. Let's assume that for each parent $x,y \in P$, $l_x$ and $l_y$ are even. There are two constructor rules.

Rule 1: $x \in P \implies [x] \in P$. Since $l_x$ is even, and the child in this case has two characters added, that means $l_{child}$ is even. So it is true for this case.

Rule 2: $x,y \in P \implies xy \in P$. Since $l_x$ and $l_y$ are even, when they are concatenated the length is $l_x + l_y$, and two even numbers added together give an even number, so $l_{xy}$ is even too.

Since both constructor rules give an even string length, every string in $P$ has even length.










\end{document}
