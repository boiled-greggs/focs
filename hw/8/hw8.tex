\documentclass{article}
\usepackage[margin=1.0in]{geometry}
\usepackage{amsmath, amssymb, mathrsfs}
\usepackage{verbatim}
\usepackage{mathtools}
\usepackage[english]{babel}
\usepackage{graphicx}
\usepackage{enumerate}
\usepackage{tikz}
\usepackage{ wasysym }
\usetikzlibrary{shapes,backgrounds}
\DeclarePairedDelimiter\Floor\lfloor\rfloor
\DeclarePairedDelimiter\Ceil\lceil\rceil

\title{Foundations of Computer Science HW 8}
\author{Greg Stewart}
\date{\today}

\begin{document}

\maketitle

\section*{Q1. \normalsize Use the pigeonhole principle.}

\begin{enumerate}[(a)]\setcounter{enumi}{5}
  \item \textit{For any set of $n$ integers, there is a subset whose sum is divisible by $n$.}

    Consider the integers $x_1, x_2, \dots, x_{n+1}$ with $x_j - x_i$ being a sum of distinct elements of the set of $n$ integers, for every $i < j$. If we have the set of $n$ integers defined as $\{y_1, y_2,\dots,y_n\}$, we can write $x_1 = 0$, $x_2 = y_1$, $x_3 = y_1 + y_2$, and so one. Now let's define $z_i = y_1 + y_2 + \cdots + y_i$ for all $0 \leq i \leq n$, where $i$ are integers. Obviously, there are $n+1$ $c_i$'s, so due to the pigeonhole principle, at least 2 must be congruent modulo $n$. Thus we can write 

    $$z_i \equiv z_j \mod n$$

    for some $i < j$. Clearly this gives us

    $$z_i - z_j \equiv y_{i+1} + y_{i+2} + \cdots + y_j \equiv 0 \mod n$$

    So there must be a subset of the given set of integers which is divisible by $n$.
\end{enumerate}





\section*{Q2. \normalsize How many times does a 1 appear in the list of integers from 0 to 9,999,999?}

Every number in this range can be expressed as a string of digits:

$$d_1d_2d_3d_4d_5d_6d_7$$

Now we can hold any combination of these digits to be 1 and the rest to not be 1 to calculate the total number of numbers where 1 appears. From basic combinations, we obtain the expression

$$7(1\cdot9\cdot9\cdot9\cdot9\cdot9\cdot9) + 21(1\cdot1\cdot9\cdot9\cdot9\cdot9\cdot9) + 35(1\cdot1\cdot1\cdot9\cdot9\cdot9\cdot9) + 35(1\cdot1\cdot1\cdot1\cdot9\cdot9\cdot9) + 21(1\cdot1\cdot1\cdot1\cdot1\cdot9\cdot9) + 7(1\cdot1\cdot1\cdot1\cdot1\cdot1\cdot9) + 1(1\cdot1\cdot1\cdot1\cdot1\cdot1\cdot1)$$

Which evaluates to 

$$5, 217, 031$$

So $5217031$ numbers in the given range contain the digit 1.

\section*{Q3. \normalsize Toss 3 fair coins.}

\begin{enumerate}[(a)]
  \item \textit{There is at least one head. What is $\mathbb{P}$[at least 2 heads]?}

    The possibilities are HHH, HHT, HTH, HTT since there is at least one head known. All of these are equal in probabilty, so if there is at least one head, then

    $$\mathbb{P}[\text{ at least 2 heads }] = 3\cdot\frac{1}{4} = \frac{3}{4}$$
  \item \textit{You know that at least two match. The third matches those matching two with probability $\frac{1}{2}$, so the probability that all three match is $\frac{1}{2}$. Do you agree?}

    This is incorrect because it's not necessarily the same two coins which match in the first place. There are $2^3 = 8$ total possible outcomes, and only two of them show all three coins matching: HHH and TTT. Thus, for the probabilties of all three matching, we have

    $$\frac{2}{8} = \frac{1}{4}$$
\end{enumerate}





\section*{Q4. \normalsize Cards are drawn randomly from a 52-card deck until a spade is drawn.}

\begin{enumerate}[(a)]
  \item \textit{What is the probability that no hearts have been drawn?}

    Let's take a look at the first few cases, where the first card drawn is a spade, then where the first card is neither heart nor spade but the second card is a spade, and so on.

    \begin{align*}
      P_1 &= \frac{13}{52} \\
      P_2 &= \frac{26}{52}\frac{13}{51} \\
      P_3 &= \frac{26}{52}\frac{25}{51}\frac{13}{50} \\
    \end{align*}

    The pattern we appear to be reaching is 

    $$\frac{\frac{26!}{(26+1-k)!}}{\frac{52!}{(52-k)!}}\cdot13$$

    Where $k = 1,2,\cdots,26$ since at most 26 cards could be drawn without drawing a spade or a heart. We can evaluate this as a sum:

    $$\sum_{k=1}^{26} \frac{26!(52-k)!}{52!(27-k)!}\cdot 13 \approx 0.5$$
\end{enumerate}

\textit{Verify the answer with Monte-Carlo.}

Using Monte-Carlo and 10,000,000 simulations, this result was confirmed as the probability that no hearts were drawn before a spade turned out to be 0.4999262.





\section*{Q5. \normalsize Randomly pick a card from a 52-card deck. What is the probability:}

\begin{enumerate}[(a)]
  \item \textit{The card is a king given it is a spade?}

    $$\mathbb{P}[king \mid spade] = \frac{1}{52}\frac{52}{13} = \frac{1}{13}$$
\end{enumerate}





\section*{Q6. \normalsize You roll two dice independently. Which pairs of events are independent?}

\begin{enumerate}[(a)]
  \item \textit{$A = \{$ first is odd $\}$ and $B = \{$ second is even $\}$}

    These are independent events. The first die being odd has no effect on whether the second die is even.
  \item \textit{$A = \{$ sum is 10 $\}$ and $B = \{$ both are odd $\}$}

    These are not independent events. We can prove this easily by looking at the conditional probabilities and checking if they match the respective single event probabilities:

    $$P[A\mid B] = \frac{1}{36}\frac{36}{3} = \frac{1}{3}$$
    $$P[B\mid A] = \frac{1}{36}\frac{36}{3} = \frac{1}{3}$$
    $$P[A] = \frac{3}{36} = \frac{1}{12}$$
    $$P[B] = \frac{3}{36} = \frac{1}{12}$$

    So obviously $P[A\mid B] \neq P[A]$ and $P[B \mid A] \neq P[B]$. Thus the events are not independent.

\end{enumerate}





\section*{Q7. \normalsize A biased die rolls values in $\{1,2,3,4,5,6\}$ with probabilities $p_1, p_2, p_3, p_4, p_5, p_6$. Can one choose $p_i$ so that the PDF of the sum of two rolls is uniform on $\{2,\dots,12\}$?}

We must first have $p_1 = p_6$, so the probability of getting 2 and 12 is the same. In fact, we can write out a system of equations based on the probabilities of getting the sums.

\begin{table}[h!]
  \centering
  \begin{tabular} {c | c}
    sum & probability \\ [0.5ex]
    \hline
    2 & $p_1p_1$\\
    3 & $2p_1p_2$\\
    4 & $2p_1p_3 + p_2p_2$\\
    5 & $2p_1p_4 + 2p_2p_3$\\
    6 & $2p_1p_5 + 2p_2p_4 + p_3p_3$\\
    7 & $2p_1p_6 + 2p_2p_5 + 2p_3p_4$\\
    8 & $2p_2p_6 + 2p_3p_5 + p_4p_4$\\
    9 & $2p_3p_6 + 2p_4p_5$\\
    10 & $2p_4p_6 + p_5p_5$\\
    11 & $2p_5p_6$\\
    12 & $p_6p_6$\\
  \end{tabular}
\end{table}

From this we see that $p_2 = \frac{p_1}{2}$. We can repeatedly substitute this value and the subsequently calculated values for $p_i$ to get all the $p_i$ in terms of $p_1$. From this we get

\begin{align*}
  p_2 &= \frac{p_1}{2} \\
  p_3 &= \frac{3p_1}{8} \\
  p_4 &= \frac{5p_1}{16} \\
  p_5 &= \frac{35p_1}{128} \\
  p_6 &= \frac{63p_1}{256}
\end{align*}

But this is only from setting the first six equations equal to one another, so that their probabilities are uniform. In fact, if we continue to do this with the rest of the probability expressions, we soon find that these values for probability don't work---if they did, they should all evaluate to $p_1p_1$. This is not the case. Take the probability we get a sum of 12:

$$p_6p_6 = \frac{63p_1}{256}\frac{63p_1}{256} = \frac{63^2}{256^2}p_1p_1 \neq p_1p_1$$

And this is indeed the case for all the others. One can pick any set of expressions to set equal to one another; this will always be the case, if $p_i \neq 0 \forall i$.

Thus, the only solution that gives a uniform PDF is

$$p_i = 0 \quad \forall i : 1 \leq i \leq 6$$

%%%%%%%%%%
\begin{comment}
Now all we need is 

$$\sum_{i=1}^6 p_i = 1$$

Which can write, using the previous results, as

\begin{align*}
  p_1 + \frac{p_1}{2} + \frac{3p_1}{8} + \frac{5p_1}{16} + \frac{35p_1}{128} + \frac{63p_1}{256} &= 1 \\
  p_1 &= \frac{1}{1} + \frac{1}{2} + \frac{3}{8} + \frac{5}{16} + \frac{35}{128} + \frac{63}{256} \\
  &= \frac{256}{693} \\
  &\approx 0.369
\end{align*}

And now that we have $p_1$, we can easily calculate the others:

\begin{align*}
  p_1 &= \frac{256}{693} \approx 0.369 \\
  p_2 &= \frac{128}{693} \approx 0.185 \\
  p_3 &= \frac{32}{231} \approx 0.139 \\
  p_4 &= \frac{80}{693} \approx 0.115 \\
  p_5 &= \frac{10}{99} \approx 0.101 \\
  p_6 &= \frac{1}{11} \approx 0.909 \\
\end{align*}

\end{comment}
%%%%%%%%%%%%%%%%%%%%%%

\section*{Q8. \normalsize Same set up for die roll. Give the CDF for the maximum out of 10 rolls of the die. From the CDF, obtain the PDF.}

\bgroup
\def\arraystretch{1.5}
\begin{table}[h!]
  \centering
  \begin{tabular} {c | c}
    CDF & probability \\ [0.5ex]
    \hline
    $F_X(6)$ & 1\\
    $F_X(5)$ & $\Big(\sum_{i=1}^5 p_i \Big)^{10}$ \\
    $F_X(4)$ & $\Big(\sum_{i=1}^4 p_i \Big)^{10}$ \\
    $F_X(3)$ & $\Big(\sum_{i=1}^3 p_i \Big)^{10}$ \\
    $F_X(2)$ & $\Big(\sum_{i=1}^2 p_i \Big)^{10}$ \\
    $F_X(1)$ & $p_1^{10}$ \\
  \end{tabular}
\end{table}

The CDF is just a cumulative sum of the probabilities, so each probability in the PDF can be obtained by subtracting from its corresponding CDF value $F(x)$ the value of $F(x-1)$.

\bgroup
\def\arraystretch{1.5}
\begin{table}[h!]
  \centering
  \begin{tabular} {c | c}
    PDF & probability \\ [0.5ex]
    \hline
    $P_X(6)$ & $1 - F_X(5)$ \\
    $P_X(5)$ & $F_X(5) - F_X(4)$ \\
    $P_X(4)$ & $F_X(4) - F_X(3)$ \\
    $P_X(3)$ & $F_X(3) - F_X(2)$ \\
    $P_X(2)$ & $F_X(2) - F_X(1)$ \\
    $P_X(1)$ & $F_X(1)$ \\
  \end{tabular}
\end{table}




\section*{Q9. \normalsize Roll 4 independent fair dice. What is $\mathbb{P}$[ exactly one 2 and one 4 ]?}

There are $6^4$ total possible outcomes, and if we hold that one die must be 2 and one must be 4, there are four die which could be 2 upon rolling, then three die which could be 4, and $4^2$ other possibilities given that the other two dice may not be 2 or 4. Thus, we get

$$\mathbb{P} = \frac{4(3)(4^2)}{6^4} = \frac{4}{27}$$





\section*{Q10. \normalsize For each PDF, show that the sum of the probabilities is 1.}

\begin{enumerate}[(a)]
  \item $P_X(k) = \frac{1}{n} \text{ for } k = 1,2,\dots,n$

    This can be written as
    
    $$\sum_{i=1}^n \frac{1}{n}$$

    Which evaluates to 

    $$n\cdot\frac{1}{n} = 1$$

    So the sum is 1.
  \item $P_X(k) = B(k;n,p) \text{ for } k = 0,1,\dots,n$

    From the definition of B(k;n,p), we can write this as

    $$\sum_{k=0}^n \frac{n!}{k!(n-k)!}p^k(1-p)^k$$

    Which evaluates to 

    $$(1-p)^n + (1-p)^n\frac{1}{(1-p)^n} - (1-p)^n = 1$$

    So the sum is clearly 1.
  \item $P_X(k) = p(1-p)^{k-1} \text{ for } k = 1,2,3,\dots$

    This sum is a geometric series, and so can be evaluated as such:

    \begin{align*}
      p\cdot \sum_{k=1}^{\infty} (1-p)^{k-1} &= p\cdot\frac{1}{1-(1-p)} \\
      &= p\cdot\frac{1}{p} \\
      &= 1
    \end{align*}

    Clearly the sum is 1.
\end{enumerate}










\end{document}
