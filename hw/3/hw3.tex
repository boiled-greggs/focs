\documentclass{article}
\usepackage[margin=1.0in]{geometry}
\usepackage{amsmath, amssymb, mathrsfs}
\usepackage[english]{babel}
\usepackage{graphicx}
\usepackage{enumerate}
\usepackage{tikz}
\usetikzlibrary{shapes,backgrounds}

\title{Foundations of Computer Science HW 3}
\author{Greg Stewart}
\date{\today}

\begin{document}

\maketitle

\section*{Q1} % 4.12 h & l
\subsection*{4.13(h) \normalsize If $n \in \mathbb{Z}$, then $n^2 - 3$ is not divisible by 4.}

Suppose that $n^2 -3$ \textit{is} divisible by 4. Then $n^2 - 3 = 4k$ for $k \in \mathbb{Z}$, so $$n^2 = 4k + 2$$ There are two cases to consider here---when $n$ is even, and when $n$ is odd.

For $n$ odd, we know that $n^2$ is odd, but the right hand side of the above inequality is even, so we immediately have a contradiction.

For $n$ even, $n = 2l$ for $l \in \mathbb{Z}$ so we end up with the expression $$4l^2 = 4k + 2$$

From the left hand side, $4l^2 \mod 4 = 0$ and from the right, $(4k+2)\mod 4 = 2$, and since these remainders are different, the equality is impossible, so we have another contradiction.

As both cases lead to contradiction, it must be true that $n^2 -3$ is not divisible by 4.

\subsection*{4.13(l) \normalsize When dividing $n$ by $d$, the quotient $q$ and remainder $0 \leq r < d$ are unique.}

Suppose there is also $q'$ and $r'$ such that $0 \leq r < b$ and $-b < -r' \leq 0$ for $a = bq + r = bq' + r'$. we can rearrange these inequalities so $-b < r-r' < b$. The two equations for $a$ can also be arranged so $bq' - bq = r-r'$, whcih means $r - r'$ is divisible by $b$. For $r-r' \neq 0$, this implies $b \leq r-r'$, which is clearly a contradiction. We must have that $r = r'$ and $q = q'$, so $r$ and $q$ are unique.


\section*{Q2} % 4.27 b & c
\subsection*{4.28(b) \normalsize $A = \{7k, k \in \mathbb{N}\}$ and $B = \{3k, k \in \mathbb{N}\}$. Prove that $A \cap B \neq \emptyset$}

In set $A$, all of the elements are multiples of 7, and in $B$, all elements are multiples of 3. $A \cap B$ contains all the elements that $A$ and $B$ share, so what must be shown is that there is at least one element that is divisible by both 7 and 3. $3 \cdot 7 = 21$, and obviously 21 is divisible by both 3 and 7, so it is contained within both sets. Thus $A\cap B$ contains, at the very least, the number 21. Thus $A\cap B \neq \emptyset$.

\subsection*{4.28(c) \normalsize $A = \{4k-3, k \in \mathbb{N}\}$ and $B = \{4k + 1, k \in \mathbb{N} \}$. Prove or disprove $A = B$.}

While in general, $4k -3$ and $4k + 1$ will give equivalent elements for $n \geq 2$, the important value for $k$ here is a base case---the first element of $\mathbb{N}$. When $k = 1$, we have in $A$ $$4(1) - 3 = 1$$ while in $B$ we have $$4(1) + 1 = 5$$. So the first element of $A$, $a_0$, is less than the first in $B$, $b_0$. Since there is a value $x \in A$ for which $x \notin B$, $A \not\subseteq B$, so $A \neq B$. 

\section*{Q3}
\subsection*{5.3(c) \normalsize $P(2)$ is \textbf{T} and $P(n) \implies (P(n^2) \wedge P(n-2))$ is \textbf{T} for $n \geq 1$.}

Using the implication, we have

$$P(2) \implies (P(4) \wedge P(0))$$

$$P(4) \implies (P(16) \wedge P(2))$$

$$P(16) \implies (P(256) \wedge P(14)$$

$$P(256) \implies (P(65536) \wedge P(254)) \text{ and } P(14) \implies (P(196) \wedge P(12))$$

As is becoming obvious, each new implication in the chain leads to every even $n$ below $n^2$ giving $P(n)$ as true. This is due to the $P(n-2)$ part of the predicate. The evenness comes from starting with $P(2)$---each $n^2$ is even, so every subsequent $n^2$ is of course even, and if $n$ is even then $n-2$ is also even. So finally, we have the conclusion that $P(n)$ is true for all $n$ even.


\section*{Q4} % 5.9 d
\subsection*{5.10(d) \normalsize Prove by induction: $3^n > n^2$.}

For the base case $n=1$, we have

$$3^1 = 3 > 1^2 = 1$$

which is obviously true. Let's assume that $3^n > n^2$ for some $n$. Then for $n + 1$ we have

\begin{align*}
  3^{n+1} &> (n+ 1)^2 \\
  3\cdot 3^n &> n^2 + 2n + 1 \\
  &> 3n^2 > n^2 + 2n + 1
\end{align*}

The last step above is true for sufficiently large $n$, which we must solve for.

\begin{align*}
  3n^2 &> n^2 + 2n + 1\\
  3n^2 - n^2 -2n -1 &> 0 \\
  2n^2 - 2n - 1 &> 0
\end{align*}

Solving for $n$ we get $n = \frac{1+\sqrt{3}}{2} \approx 1.4$, so this is true for all $n \geq 2$, and since we have the base case as true, we can say that $3^n > n^2$ for all $n \geq 1$.

\section*{Q5}
\subsection*{5.14 \normalsize Prove by induction that every $n \geq 1$ is a sum of distinct powers of 2.}

Take the base case, $n = 0$. $0\cdot 2^0 = 0 = n$ so this is true for the base case. It's clearly also true for $n = 1$: $2^0 = 1 = n$. Assume that for some $n$, for all $m$ such that $0 \leq m \leq n$, $P(m)$ is true---$m$ can be written as a sum of distinct powers of 2. 

Consider the $n+1$ case. Let $2^k$ be the largest power of 2 less than or equal to $n+1$. $2^k \geq 1$ for any $k \in \mathbb{N}$, so $$n + 1 - 2k \leq n + 1 - 1 = n$$, which, using the assumption, means that $n + 1 - 2^k$ is also the sum of distinct powers of 2. If $A$ is the set of these powers of 2, and $a_i$ are the elements in $A$, then $$n + 1 = 2^k + \sum_{i} a_i$$

Since all the powers of 2 in $A$ are distinct by the assumption, it must be shown that $2^k \notin A$. Assume that $2^k \in A$. Then the sum of powers of 2 in $S$ is $n + 1 - 2^k$, which leads to

\begin{align*}
  2^k &\leq n + 1 - 2^k \\
  2\cdot 2^k &\leq n + 1 \\
  2^{k+1} \leq n+1
\end{align*}

So $2^k$ is \textit{not} the largest power of 2 which is less than or equal to $n+1$, so $2^k \notin A$. Therefore $n+1$ can be expressed as a sum of distinct powers of 2, so $P(n) \implies P(n+1)$ and any natural number can be written as a sume of distinct powers of 2.


\section*{Q6}
\subsection*{5.16 \normalsize Let $A$ be a finite set of size $n \geq 1$; prove by induction that $|\mathscr{P}(A)| = 2^n$}

For the base case, take $|A| = 0$, so $A = \emptyset$, and the empty set is its only subset, so $|\mathscr{P}(A)| = 1 = 2^0$.

Assume that for $|A| = n$, it is true that $|\mathscr{P}(A)| = 2^n$. Let $B$ be a set with one more element than $A$, so we can say $B = A \cup \{b\}$. The subsets of $B$ can be separated into two categories---some include $b$, and some do not. In the latter case, they are exactly the subsets of $A$, so there are $2^n$ of these. The former is all of these subsets plus $b$, which we can write as $C \cup {b}$ where $C \in \mathscr{P}(A)$. There are $2^n$ possibilities for $Z$, so so there are $2^n$ examples of $Z \cup {b}$. Thus,

$$|\mathscr{P}(B)| = 2^n + 2^n = 2\cdot 2^n = 2^{n+1}$$

So by induction it's true that $|\mathscr{P}(A)| = 2^n$.


\section*{Q7} % 5.29
\subsection*{5.30 \normalsize Prove you can make any postage greater than 12 cents using only 4 and 5 cent stamps.}

We need a few base cases here. $12 = 3\cdot 4 + 0 \cdot 5$. $13 = 2\cdot 4 + 1\cdot 5$. $14 = 1\cdot 4 + 2\cdot 5$. $15 = 0\cdot 4 + 3 \cdot 5$. 

Okay, let's assume that for $n \geq 15$, we have that $$k\cdot4 + l\cdot5 = n$$

For $n+1$, there are two cases. In the first case, $k > 0$, so all we need to do is subtract one from $k$ and add one to $l$, i.e.

$$(k-1)\cdot 4 + (l+1)\cdot 5 = n+1$$

For the $k = 0$ case, we must subtract 3 from $l$ and add 4 to $k$ to get the next number, so

$$(k+4)\cdot 4 + (l-3)\cdot 5 = n + 1$$

So $P(n) \implies P(n+1)$ for all $n \geq 12$.



\section*{Q8} % 5.43
\subsection*{5.47(a) \normalsize Robot moves one diagonal step at a time. Prove no move sequence can move the robot one square to the right of its initial position.}

Let $(x,y)$ be the position of the robot on the grid, and let $(0,0)$ be its starting position. A brief inspection reveals that after the first step, the robot's position could be any of $(1,1)$, $(1,-1)$, $(-1,1)$, or $(-1,-1)$. In the next step, there are of course more possible positions, including back at $(0,0)$, but none of these are the position $(1,0)$, to the right of the intiial position. Going in any direction for more steps seems to reveal a relationship between $x$ and $y$---$x+y$ is always even, and we call this predicate $P(n)$. We prove this by induction.

The $n = 0$ step is true because for $P(0)$ we have $0 + 0 = 0$, which is even. Now assume that $P(n)$ is even, so we know $x+y$ is even. Consider $P(n+1)$. There are four possible moves for the robot, so there are four cases to look at.

First, $P(n+1) = (x+1, y+1)$. The sum is $x+1+y+1 = x+y+2$ and since $x+y$ is even, the sum is even too, so $P(n+1)$ is even.

For the move to $(x+1, y-1)$, the sum is just $x+y$, which we know is even, so $P(n+1)$ is even.

Now for $(x-1, y+1)$, the sum is again just $x+y$ so $P(n+1)$ is even.

Finally, for $(x-1,y-1)$, the sum is $x+y-2$, which is even, so $P(n+1)$ is even.

In all the cases, $P(n+1)$ is even, so the robot can never be on a square whose coordinates sum to an odd number, as is the case with $(1,0)$, so it's impossible for the robot to ever end up there.

\subsection*{5.47(b) \normalsize A move changed! The (x+1,y+1) move is now (x+1, y+2). Prove the robot can go to any coordinate $(m,n)$.}

Let's call the predicate $P(k): $ the robot can reach any square $(m,n)$ in a finite sequence of moves. By inspection, one can see that all of the squares surrounding the initial position are reachable by the robot in a finite number of moves, and these can be shown easily. The formula for the next location $(m,n)$ is 

$$(m,n) = a(x+1,y+2) + b(x+1, y-1) + c(x-1, y+1) + d(x-1,y-1)$$

So aside from the trivial cases, we have surrounding the initial position:

\begin{align*}
  (1,0) &\implies (a,b,c,d) = (1,1,1,0) \\
  (0,-1) &\implies (a,b,c,d) = (1,1,2,0) \\
  (-1,0) &\implies (a,b,c,d) = (1,0,2,0) \\
  (0,1) &\implies (a,b,c,d) = (1,0,1,0) \\
  (1,1) &\implies (a,b,c,d) = (2,1,2,0) 
\end{align*}

What this means is that, from any position on the grid, a square adjacent or diagonally adjacent to that position can be reached in a finite number of moves. Since all eight surrounding squares can be reached from the $(m,n)$ position, this means that, in the next move, any surrounding square can be reached from the $(m+1, n+1)$ position in a finite number of moves. As every step only involves a finite number of moves, to reach any square only a finite number of moves is needed, so any square can be reached from the origin by a finite sequence of moves.






























\end{document}
